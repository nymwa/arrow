\subsection{多数決決定法の強み}

今まで多数決決定法について色々見てきました.実は多数決決定法にしかできないことというのがあって,この節ではそれについて述べたいと思います.

まず,今まで社会的厚生関数や社会的決定関数と言っていたものをもう少し一般化します.

\begin{dfn}[集合的選択規則]
    集合的選択規則とは,n人の個人的選好の弱順序の任意の組$(P_1,\dots,P_n)$に対して,社会的選好の二項関係をただ1つに割り当てる関数関係のことです.
\end{dfn}

社会的厚生関数では社会的選好が弱順序でしたが,集合的選択規則ではその制限はありません.しかし,同じ選択肢同士は少なくとも同程度以上にみなすべきなので,反射性は成り立つものとしてよいでしょう.

\begin{dfn}[決定力]
    集合的選択規則に決定力があるとは,その値域が完全性を持つ二項関係に限定されているということです.
\end{dfn}

集合的選択規則による社会的選好に完全性も求めるということです.社会的厚生関数は推移性を満たす決定力のある集合的選択規則,社会的決定関数は非循環性を満たす決定力のある集合的選択規則であると言えるでしょう.

\begin{lem}
    無関係選択肢からの独立性を満たす任意の決定力がある集合的選択規則の$A$の異なる2選択肢$x,y$に関する社会的選好は,$x,y$に関する個人的選好の関数でなければならない.
\end{lem}

これは無関係選択肢からの独立性の定義より明らかです.

次に集合的選択規則に関する3つの条件を定義します.この3つの条件の定義において,集合的選択規則$f$は,個人的選好のn組$(P^1_1,\dots,p^1_n)$と$(P^2_1,\dots,P^2_n)$をそれぞれ社会的選好$\succeq^1$と$\succeq^2$に写像するものとします.

\begin{dfn}[個人の平等性]
    個人的選好のn組$(P^1_1,\dots,P^1_n)$を並び変えて$(P^2_1,\dots,P^2_n)$としたとき,すべての選択肢$x,y$に対して$x \succeq^1 y \leftrightarrow x \succeq^2 y$を満たすとき,$f$は個人の平等性を満たすと言います.
\end{dfn}

これは,たとえばn組の一番始めの人が独裁者になるというような$f$は個人の平等性を満たさないということです.これは$f$が投票用紙に名前を書かない,誰が投票したかわからないという投票方式だということであり,実質的に1人1票,個人の平等性を意味します.

\begin{dfn}[選択肢の平等性]
    すべての4選択肢$x,y,z,w$に対して,
    \begin{equation*}
        \forall i (x \geq^1_i y \leftrightarrow z \geq^2_i w) \land \forall i (y \geq^1_i x \leftrightarrow w \geq^2_i z)
        \to (x \succeq^1 y \leftrightarrow z \succeq^2 w) \land (y \succeq^2 x \leftrightarrow w \succeq^2 z)
    \end{equation*}
    の時,$f$は選択肢の平等性を満たすと言います.
\end{dfn}

なんだかよくわからない式が出てきました.この式の意味をちょっと考えてみたいと思います.

前件のところを見てください.これは,選択肢$x,y,z,w$を取ってきたときに「選好1でのxとyの関係,選好2でのzとwの関係が同じ時について考えましょう」という意味です.そして,後件を見ると,「xとyの社会的選好1とzとwの社会的選好2も同じになる」とあります.

例えば,今xとzが5票,yとwが3票を獲得したとします.このとき社会的選好1では「xはyより良い」となったのに,2の方では「wのほうがzより良いです」となってしまったのでは,選択肢ごとに1票が与える価値が異なってきてしまいます.この時選択肢の平等性は,「zはwより良い」となるというのを要請します.このように,選択肢の平等性は全く同じ選好状況に対しては全く同じ結論にならなければならないということを要請します.

さらに,選択肢の平等性は選択肢を入れ替えても成り立っていなければなりません.これは,選択肢の名前などではなく,選択肢そのものやその質的なもの,内実に対して評価をして選択をしているということを保証します.

もしも選択肢の平等性がないと,社会的選好に対する信頼性が失われることは容易に想像がつきます.先程の例でzでなくwが社会的に選ばれたらzに投票した人を説得することは難しいでしょう\footnote{ただし,この章の最後でこの条件を否定しにかかりますが.}.

\begin{dfn}[正の反応性]
    すべての2つの選択肢$x,y$に対して,
    \begin{equation*}
        \begin{array}[h]{l}
            [\forall i ((x >^1_i y \to x >^2_i y) \land (x \doteq^1_i y \to x \geq^2_i y)) \land \\
            \exists j ((x \doteq^1_j y \land x >^2_j y) \lor (x <^1_j y \land x \geq^2_j y))
            ] \to (x \succeq^1 y \to x \succ^2 y)
        \end{array}
    \end{equation*}
    の時,$f$は正の反応性を持つと言います.
\end{dfn}

なにがなんなんだっていう感じの式が出てきました.この条件はそんなに大事じゃない(気がする)ので,「そして」から始まる段落まで読み飛ばして頂いても構いません.

$\forall i ((x >^1_i y \to x >^2_i y) \land (x \doteq^1_i y \to x \geq^2_i y))$の意味を考えてみます.これは,任意の個人$i$は,選好1で「xはyより良い」と思ってたら選好2でも「xはyより良い」のままで,選好1で「xとyは同じ」と思っていたら選好2では「xとyは同じ」のままか「xはyより良い」となっているかのどちらかという意味です.つまり,予めxをyより少なくとも同じくらい良いと評価していた人は,xとyの評価をyを良いとする方向には変えないという状況です.

$\exists j ((x \doteq^1_j y \land x >^2_j y) \lor (x <^1_j y \land x \geq^2_j y))$は,ある個人$j$が,選好1で「xとyは同じ」と思っていたら,選好2では「xはyより良い」となる,または選好1で「yはxより良い」と思っていたら,選好2では「xとyは同じ」か「xはyより良い」となります.つまり,少なくとも誰か1人がもともとyをxより少なくとも同程度には良いと考えていたけれど,xに優位になる方に評価を変えたという状況です.

そして,$x \succeq^1 y \to x \succ^2 y$は,社会的選好1と比べて2では,xとyの評価をyが良いとする方向には変化しないとなります.ここから,正の反応性が要請するのは,xを優位としている人は皆xを優位にしたままで,少なくとも1人が選好をxに優位に変化させた時,社会的選好がもともとxとyを同じとしていたならば,社会的な選好もxを優位にする方向に変化するという意味です.

この3つの性質から多数決決定法に関する衝撃的な事実が導かれて行きます.

\begin{lem}
    個人の平等性を満たす集合的選択規則は,非独裁制です.
\end{lem}

独裁制を満たす場合,ある個人の選好が社会的にも反映されますが,その場合,個人を入れ替えた時の社会的選好が変わることから対偶が真なので,明らかです.また,独裁者がいることが個人の平等性と反するということからも明らかだと思います.

この補題は逆が成立しません.たとえ非独裁制でも,「30歳以上は1人2票,30歳未満は1人1票」というような投票方式は個人の平等性を満たしていません.このように,個人の平等性は非独裁制の条件を厳しくしたものだと言えます.

\begin{lem}
    選択肢の平等性を満たす集合的選択規則は,無関係選択肢からの独立性を満たします.
\end{lem}

選択肢の平等性の式の$z,w$に$x,y$を代入すると,無関係選択肢からの独立性の式と同じになります.これもまた逆は成り立ちません.よって,選択肢の平等性は無関係選択肢からの独立性の条件を厳しくしたものだと言えます.

\begin{lem}
    選択肢の平等性と正の反応性を満たす集合的選択規則は,社会の決定性を満たします.
\end{lem}

選択肢の平等性より$\forall i (x \doteq_i y) \to x \simeq y$です\footnote{zにy,wにxを代入します.}.これと,正の反応性より$\exists i (x >_i y) \land \forall i (x \geq_i y) \to x \succ y$なので,結局,社会の決定性$\forall i (x >_i y) \to x \succ y$は当然満たされます.

こう見ると,アロー4条件は案外弱いものだったということに気づいてもらえると思います.

\begin{thm}
    個人的選好の自由,個人の平等性,選択肢の平等性,正の反応性を持つことは,決定力のある集合的選択規則が多数決決定法となるための必要十分条件です.
\end{thm}

多数決決定法は社会的厚生関数ではありません.しかし,反射性と完全性を満たし,アロー4条件を厳しくした上の3つの条件をすべて満たし,かつ個人的選好の自由を保証するような集合的選択規則は多数決決定法しかないのです.

\begin{proof}
必要性は簡単です.多数決決定法が{\bf 決定力のある集合的選択規則の制約下}で個人的選好の自由は,任意の2選択肢$x,y$に対して$x \succeq y$または$x \preceq y$であることから満たされます.個人の平等性,選択肢の平等性を満たすことも明らかです.多数決決定法では,誰かが$x,y$の$x$の方に優位に選好を変えた時,社会的に$y$に優位に傾くことはないので,正の反応性も満たされます.

十分性を示します.補題5.3より,無関係選択肢からの独立性を満たし,補題5.1より,2選択肢$x,y$に関する社会的選好は各個人の$x,y$に関する社会的選好のみによって定まります.また,個人の平等性より,$x,y$に関する社会的選好は「xをyより好むもの」「yをxより好むもの」「xをyと同程度に好むもの」の人数にのみ依存しなければなりません.

選択肢の平等性より,$N(x > y) = N(x < y) \to x \simeq y$です.というのも,$N(x > y) = N(x < y)$を仮定し,$x$と$y$を入れ替えただけで個人の選好が変わっていない個人的選好$>'$で,$N(x \geq y) = N(y \geq' x), N(y \geq x) = N(x \geq' y)$です.

選択肢の平等性で,zにy,wにxを代入すると,$(x \succeq y \leftrightarrow y \succeq' x) \land (y \succeq x \leftrightarrow x \succeq' y)$となります.

$N(x \geq y) = N(y \geq x) = N(x \geq' y)$,同様に$N(y \geq x) = N(y \geq' x)$であり,個人の平等性より,$\succeq$と$\succeq'$は同じ選好にならなければなりません.

よって,$x \succ y$と仮定すると,$x \succ y \to x \succ' y \to y \succ x$となり,矛盾します.$y \succ x$の時も同様にして,結局,$x \simeq y$です.

$N(x > y) = N(x < y) \to x \simeq y$が成り立つことがわかったので,正の反応性より,$N(x > y) \geq N(x < y) \to x \succeq y$です.これは多数決決定法の定義そのものに他なりません.
\end{proof}

アロー4条件をすこし強めて社会的選好の制約を弱めると,多数決決定法のみが条件を満たすのです.多数決決定法は,循環を簡単に起こせるため社会的厚生関数や社会的決定関数にはなりえません.しかし,結果がどうなるかというのを度外視してみれば,我々の日常的な感覚と同じく,公平公正な投票方式と言えるだけでなく,多数決決定法は他の投票方式よりも優れている点が多いと言えるでしょう.

