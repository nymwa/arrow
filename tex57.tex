\subsection{結局}
さて,今までアロー4条件が妥当かについて議論してきました.正直,議論は不十分極まりないという感じですが,どの条件をどこまで破ってもいいのかという点に着目しながら決定方式を議論できると言えなくはない程度のことはしたと思っています.

少し5章全体をまとめたいと思います.そもそも,社会的厚生関数の前提がダメなのではないかということで,推移性を非循環性まで緩めてあげてみると,アロー4条件を満たし得ることを示しました.しかし,あまり良い社会的決定ができる方法にはなりませんでした.

次に具体的にアロー4条件を緩めて検討しました.個人的選好の自由を制限すると社会的選好の推移性は保たることを示しました.しかし,社会的厚生関数となるために必要十分な制限である価値制限を満たすことをどう捉えるかについては,私の力不足ではありますがよくわかりませんし,制限が妥当なものとも言えませんでした.逆に,個人的選好を制限せず,社会的に反射性と完全性だけを満たしていれば良いとすれば,多数決決定法はアロー4条件を厳しくした条件もクリアするかなり優秀な方式と言えることもわかりました.

社会の決定性は場合によっては個人の権利を侵害しうるということを指摘しました.この発想から,社会的決定に倫理性という概念を持ち込むことができることも少し述べました.

無関係選択肢の独立性は,社会が開かれたものであるべきだとするハンソンの定理の要請から,別にそんなに重要な条件でもないと捉えることもできました,単記投票やボルタ得点法などは戦略的投票が可能になってしまいますが,そのようなことを各個人が前提とした社会的決定であれば民主的と言えなくもありません.

非独裁制についてはあまり詳しく述べませんでしたが,それは個人の平等性という条件に含まれるということは見ました.個人の平等性は「一人一票」のように,各個人が投票に関しては同じ権利を持つという条件なので,そう考えると非独裁制はあまり疑う余地がないように個人的には思います\footnote{実際どうなのかは詳しい人教えてください.}.

さて,次に架空世界での投票方式について考えていきたいと思います.

