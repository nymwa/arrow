\section{アローの不可能性定理の証明}
\begin{thm}[アローの不可能性定理]\label{thm:41}
    選択肢が3つ以上あり,個人が2人以上いる時,個人的選好の自由,社会の決定性,無関係選択肢からの独立性,非独裁制のすべてを満たす厚生関数は存在しません.
\end{thm}

残念ながら,「民主制に不可欠」として提示された4条件のすべてを満たす完璧な投票方式は存在しません.どのような投票方式もどこかしらに欠陥があるということです.

では,証明しましょう.以下,選択肢は3つ以上,個人は2人以上であるものとします.

\begin{dfn}[決定的集合]
    $M$の部分集合$V$が,$M$のどのような個人的選好に対しても
    \begin{equation*}
        \forall i \in V ( x >_i y) \to x \succ y 
    \end{equation*}
    を満たすとき,$V$は$x \succ y$と選好させる決定的集合と言います.$V$が$x \succ y$とさせる決定的集合であることを$D_V(x \succ y)$と表します.
\end{dfn}

決定的集合というのは,ある選択肢のペア,例えば「ラーメンとカレー」についてその集合の中の全員が「ラーメンはカレーより良い」と意見が一致すれば,その集合以外の誰がどう選好しようと社会的選好を「ラーメンはカレーより良い」と{\bf 決定させることができる}"決定力"のある集合です.

\begin{dfn}[決定力]
    $D_V(x \succ y)$の時,$V$は$x \succ y$とする決定力があるといいます.また,$D_V(x \succ y) \land D_V(x \prec y)$の時,$V$は$x,y$に対して決定力があるといいます.
\end{dfn}

決定的集合には決定力があります.

\begin{dfn}[準決定的集合]
    $M$の部分集合$V$が,$M$のどのような個人的選好に対しても
    \begin{equation*}
        \forall i \in V ( x >_i y ) \land \forall j \not \in V ( x <_i y) \to x \succ y
    \end{equation*}
    を満たすとき,$V$は$x \succ y$と選好させる準決定的集合と言います.$V$が$x \succ y$とさせる準決定的集合であることを$S_V(x \succ y)$と表します.
\end{dfn}

準決定的集合は少し変わっています.というのも,$V$の中の全員が「ラーメンはカレーより良い」と言うだけでなく,$V$以外の全員が「カレーはラーメンより良い」と言っているときは,社会的にも「ラーメンはカレーより良い」になるという少し変わった力を持った集合だからです.現実的にはこのような決定力を持つ集団はいないのでしょうが,証明には非常に役に立ちます.

また,個人$i$のみの集合$\{i\}$が決定的集合,準決定的集合の時,本来は$D_{\{i\}}(x \succ y)$と書くべきところを略して$D_i(x \succ y)$と表記します.

\begin{lem}\label{lem:41}
    ある個人の集合$V$に対して,$D_V(x \succ y) \to S_V(x \succ y)$.
\end{lem}

\begin{proof}
$D_V(x \succ y)$と仮定します.この時,$V$の全員が$x > y$と選好するならば$x \succ y$となります.

つまり,$V$以外の全員が$x < y$と選好した時も,$V$の全員が$x > y$と選好するならば$x \succ y$となります.よって$S_V(x \succ y)$.
\end{proof}

この補題は,決定的集合は準決定的集合でもあるということを主張しています.ただし,準決定的集合は決定的集合だとは限りません.

\begin{lem}\label{lem:42}
    個人的選好の自由,社会の決定性,無関係選択肢からの独立性を満たす厚生関数では,ある選択肢のペア$x,y$に対して,$S_i(x \succ y)$となる個人$i$が存在します.
\end{lem}

これは,上の2つの条件を満たす厚生関数には,1人だけの準決定的集合が必ずある,すなわち,ある1人が「ラーメンはカレーより良い」と言っていて,他の全員が「カレーはラーメンより良い」と言ったとしても,社会的に「ラーメンはカレーより良い」となるような選択肢のペアが必ず存在するということです.

\begin{proof}
社会の決定性より,社会の全員が$x > y$と選好するならば,$x \succ y$です.よって,社会は$x \succ y$とする決定的集合となります.$D_M(x \succ y)$なので,補題\ref{lem:41}より,$S_M(x \succ y)$.よって,社会の部分集合には必ず少なくとも1つ準決定的集合が存在することがわかります.

次に,$M$の部分集合から,最も人数の少ない準決定的集合$V$を用意します.すなわち,$S_V(x \succ y)$です.

もし,$V$の人数が1人ならば,その$V$が1人だけの準決定的集合となるので証明が終了します.以下では$V$が2人以上と仮定します.

ここで,$V$から個人$i$を取り出し,$V$から$i$を除いた集合を$V'$とします.$V$以外の個人の集合を$W$とします.以降の議論は$W$が空集合であっても成立することに注意してください.

個人的選好の自由より,$i$,$V'$,$W$の$x,y$に対する以下のような個人的選好についても社会的選好が1通り,この場合は$S_V(x \succ y)$より$x \succ y$に定まります.

\begin{figure}[!h]
    \centering
    \begin{tabular}{|c|c|c|c|c|} \hline
        & $i$ & $V'$ & $W$ & 社会 \\ \hline
        $x,y$ & $x > y$ & $x > y$ & $x < y$ & $x \succ y$ \\ \hline
    \end{tabular}
    \caption{x,yの選好}
\end{figure}

ここで,新たな選択肢$z$を導入します.個人的選好の自由の条件より,$z$のどのような個人的選好に対しても,社会的選好が1通りに定まります.ここで,選択肢$x,z$間と$y,z$間の選好をそれぞれ図6のように定めます.

\begin{figure}[!h]
    \centering
    \begin{tabular}{|c|c|c|c|c|} \hline
        & $i$ & $V'$ & $W$ & 社会 \\ \hline
        $x,y$ & $x > y$ & $x > y$ & $x < y$ & $x \succ y$ \\ \hline
        $y,z$ & $y > z$ & $y < z$ & $y > z$ & $ ? $ \\ \hline
        $x,z$ & $x > z$ & $x < z$ & $x < z$ & $ ? $ \\ \hline
    \end{tabular}
    \caption{x,y,zの選好}
\end{figure}

社会的選好で"?"となっているところは,今までの議論からではその選好を限定できないことを表しています.

今仮に$y \prec z$と仮定します.すると,定義より$S_{V'}(y \prec z)$となります.なぜならば,無関係選択肢からの独立性より$y,z$間の社会的選好は$y,z$間の個人的選好のみによって決まり,$y,z$間の各個人的選好が図6の例と異なっていた時,$V'$が準決定的集合であるための条件の前件が偽になり,命題自体が真になるため条件に全く影響しないからです.

しかし,今$V$が最小の準決定的集合であるはずなのに,$V$より1人少ない$V'$も準決定的集合となってしまい,矛盾します.背理法より,$y \succeq z$とわかります.

今,$x \succ y$ かつ $ y \succeq z$であることがわかりました.$z \succeq x$と仮定すると,社会的選好は弱順序なので推移律が成立し,$y \succeq z \land z \succeq x$より$y \succeq x$ですが,$x \succ y$と矛盾するので背理法より$z \not \succeq x$.さらに,完全性より$x \succ z$とわかります.すると,$x,y,z$の選好は図7のようになるとわかります.

\begin{figure}[!h]
    \centering
    \begin{tabular}{|c|c|c|c|c|} \hline
        & $i$ & $V'$ & $W$ & 社会 \\ \hline
        $x,y$ & $x > y$ & $x > y$ & $x < y$ & $x \succ y$ \\ \hline
        $y,z$ & $y > z$ & $y < z$ & $y > z$ & $y \succ z$または$y \simeq z$ \\ \hline
        $x,z$ & $x > z$ & $x < z$ & $x < z$ & $x \succ z$ \\ \hline
    \end{tabular}
    \caption{x,y,zの選好}
\end{figure}

すると,図7で先程と同様に無関係選択肢からの独立性から$S_i(x \succ z)$となります.これは$V$が最小の準決定的集合であるという仮定と矛盾します.背理法より,最小の準決定的集合は,2人以上ではありません.すなわち,個人的選好の自由,社会の決定性,無関係選択肢からの独立性を満たす厚生関数には,1人だけからなる準決定的集合が存在します.
\end{proof}

1人だけの準決定的集合が生まれました.この人は独裁者の卵です.WAKWAKの気持ちを忘れずに,個人$i$を独裁者へと育て上げましょう.

\begin{lem}\label{lem:43}
    個人的選好の自由,社会の決定性,無関係選択肢からの独立性を満たす厚生関数では,任意の異なる選択肢の組$x,y,z$と個人$i$に対して,
    \begin{align*}
        S_i(x \succ y) \to D_i(x \succ z) \\
        S_i(x \succ y) \to D_i(z \succ y)
    \end{align*}
    が成立します.
\end{lem}

$x,y$の社会的選好を決定できる個人が,関係ない選択肢$z$の社会的選好も決定できてしまうという,なんともよくわからないことが成り立ってしまいます.ここで準決定的集合の定義が活きてきます.そして独裁者の卵はこの補題を武器に独裁者へと成り上がるのです.

\begin{proof}
    $S_i(x \succ y)$を仮定します.また,個人$i$以外の全員の集合を$V$とします.

    今,個人的選好の自由の条件より,$z$のどのような個人的選好に対しても,社会的選好は1通りに定まります.そこで,$x >_i y$,$y >_i z$,$x <_V y$,$y >_V z$と仮定します.すると,個人と社会の選好は図8のようになります.
\begin{figure}[!h]
    \centering
    \begin{tabular}{|c|c|c|c|} \hline
        & $i$ & $V$ & $社会$ \\ \hline
        $x,y$ & $x > y$ & $x < y$ & ? \\ \hline
        $y,z$ & $y > z$ & $y > z$ & ? \\ \hline
        $x,z$ &    ?    &    ?    & ? \\ \hline
    \end{tabular}
    \caption{x,y,zの選好}
\end{figure}

$?$は,選好が限定できていないところです.

さて,このとき,$S_i(x \succ y)$,$x >_i y$,$x <_V y$より,$x \succ y$です.社会の全員の選好が$y > z$なので,社会の決定性より$y \succ z$です.$z \succeq x$と仮定すると,推移性より$y \succeq x$ですが,$x \succ y$と矛盾するので,背理法より$z \not \succeq x$で,完全性より$x \succ z$です.

これを表にすると図9のようになります.

\begin{figure}[!h]
    \centering
    \begin{tabular}{|c|c|c|c|} \hline
        & $i$ & $V$ & 社会 \\ \hline
        $x,y$ & $x > y$ & $x < y$ & $x \succ y$ \\ \hline
        $y,z$ & $y > z$ & $y > z$ & $y \succ z$ \\ \hline
        $x,z$ & $x > z$ &    ?    & $x \succ z$ \\ \hline
    \end{tabular}
    \caption{x,y,zの選好}
\end{figure}
$V$の$x,z$間の選好は$V$の各個人ごとにことなっていても良いことに気をつけてください.

ここで,$S_i(x \succ y)$と,$x,y$間と$y,z$間に関する個人的選好の仮定のみから,$V$の$x,z$間の選好にかかわらず個人$i$の選好が社会的に反映される,すなわち,$D_i(x \succ z)$であるということが導けました.

次に,「$x >_i y$,$y >_i z$,$x <_V y$,$y >_V z$」という仮定がなくても$D_i(x \succ z)$であることを示します.

無関係選択肢からの独立性より,選択肢のペア$x,y$間と$y,z$間に関する各個人の個人的選好にどのような変更があったとしても,$x,z$間に関する個人的選好の変化がなければ,$x,z$間の社会的選好にも変化はないはずです.そこで,$x,y$間と$y,z$間の個人的選好を考えられるすべてのパターンに変化させることを考えます.すると,図10のようになります.

\begin{figure}[!h]
    \centering
    \begin{tabular}{|c|c|c|c|} \hline
        & $i$ & $V$ & $S$ \\ \hline
        $x,y$ &    ?    &    ?    &    ?    \\ \hline
        $y,z$ &    ?    &    ?    &    ?    \\ \hline
        $x,z$ & $x > z$ &    ?    & $x > z$ \\ \hline
    \end{tabular}
    \caption{x,y,zの選好}
\end{figure}
図10より,$x,y$間,$y,z$間の個人的選好にかかわらず,$D_i(x \succ z)$となることがわかり,$S_i(x \succ y) \to D_i(x \succ z)$です.

次に,「$x <_i z$,$x >_i y$,$x <_V z $,$x <_V y$」と仮定します.$S_i(x \succ y)$などから,先程の例と同様にして,選好は図11のようになります.
\begin{figure}[!h]
    \centering
    \begin{tabular}{|c|c|c|c|} \hline
        & $i$ & $V$ & $S$ \\ \hline
        $x,y$ & $x > y$ & $x < z$ & $x \succ y$ \\ \hline
        $y,z$ & $y < z$ &    ?    & $y \prec z$ \\ \hline
        $x,z$ & $x < z$ & $x < z$ & $x \prec z$ \\ \hline
    \end{tabular}
    \caption{x,y,zの選好}
\end{figure}

先程と同様にして,$S_i(x \succ y) \to D_i(z \succ y)$とわかります.
\end{proof}

そして,補題\ref{lem:42}補題\ref{lem:43}から定理\ref{thm:41}が証明できます.

\begin{proof}
個人的選好の自由,社会の決定性,無関係選択肢からの独立性のすべてを満たす厚生関数は非独裁制を満たさないことを示せば,これら4つすべてを満たす厚生関数が存在しないと言えるので,その方針で証明します.

個人的選好の自由,社会の決定性,無関係選択肢からの独立性と補題\ref{lem:42}より,ある個人$i$が存在しある異なる選択肢$x,y$で$S_i(x \succ y)$です.

個人的選好の自由,社会の決定性,無関係選択肢からの独立性と補題\ref{lem:43}より,$x,y$と異なる任意の選択肢$z$で,
\begin{align}
    S_i(x \succ y) \to  D_i(x \succ z) \\  
    S_i(x \succ y) \to  D_i(z \succ y)  
\end{align}
となります.集合$\{i\}$を$i$と略しています.さらに,x,y,zを適当に入れ替えると,
\begin{align}
    S_i(x \succ z) \to  D_i(y \succ z) & \hspace{3em} ((2)でy \leftrightarrow z) \\  
    S_i(y \succ z) \to  D_i(y \succ x) & \hspace{3em} ((1)で(x,y,z) \to (y,z,x)) \\
    S_i(y \succ x) \to  D_i(z \succ x) & \hspace{3em} ((2)でx \leftrightarrow y) \\
    S_i(z \succ x) \to  D_i(z \succ y) & \hspace{3em} ((1)で(x,y,z) \to (z,x,y)) \\
    S_i(z \succ y) \to  D_i(x \succ y) & \hspace{3em} ((2)でx \leftrightarrow z) 
\end{align}
となります.適当に入れ替えても,補題\ref{lem:43}の議論を追えば良いので問題はありません.さらに,
\begin{align}
    S_i(x \succ y)
        & \to D_i(x \succ z) & (1)より \\
        & \to S_i(x \succ z) & 補題\ref{lem:41}より \notag \\
        & \to D_i(y \succ z) & (3)より \\
        & \to S_i(y \succ z) & 補題\ref{lem:41}より \notag \\
        & \to D_i(y \succ x) & (4)より \\
        & \to S_i(y \succ x) & 補題\ref{lem:41}より \notag \\
        & \to D_i(z \succ x) & (5)より \\
        & \to S_i(z \succ x) & 補題\ref{lem:41}より \notag \\
        & \to D_i(z \succ y) & (6)より \\
        & \to S_i(z \succ y) & 補題\ref{lem:41}より \notag \\
        & \to D_i(x \succ y) & (7)より 
\end{align}
となります.よって(8),$\dots$,(13)より,
\begin{align}
    S_i(x \succ y) \to
      & D_i(x \succ y) \land
        D_i(x \succ z) \land
        D_i(y \succ x) \notag \\ \land
      & D_i(y \succ z) \land
        D_i(z \succ x) \land
        D_i(z \succ y) \tag{*}
\end{align}
となります.今$S_i(x \succ y)$なので,個人$i$は$x$と$y$を含む任意の3選択肢の組に対して決定力を持ちます.

今,選択肢の集合$A$の要素数が3の時は,これで非独裁性の条件を満たさないため証明が終了します.次に$A$の要素数が4以上の時について考えます.

$A$から$x$と$y$以外の任意の選択肢$u,v$を取ってきます.ここで,$D_i(u \succ v)$かつ$D_i(v \succ u)$であることを示せれば,個人$i$が独裁者であることがわかります.$x,y$間,$x,y$のいずれか片方と$x,y$以外の任意の選択肢間での決定力はすでに示したので,$u,v$のいずれも$x,y$とは異なる時のみを調べれば十分です.

まず,(*)式より,$S_i(x \succ y) \to D_i(x \succ u)$がわかります.ここから,
\begin{align*}
    D_i(x \succ u)
        & \to S_i(x \succ u) & 補題\ref{lem:41}より \\
        & \to D_i(u \succ v) \land D_i(v \succ u) & (*)より
\end{align*}
最後の式変形では,(*)式を,その変数を$x,y,z$から$x,u,v$に変えて用いました.このことから,個人$i$は2選択肢$u,v$に対しても決定力を持つことがわかります.

そして,すべての2選択肢のペアに対して,この結果を適用していけば,結局$S_i(x \succ y)$という条件だけから,すべての選択肢のペア$u,v$に対して,$D_i(u \succ v)$であることがわかります.今$S_i(x \succ y)$なのですから,個人$i$は独裁者ということになります.

以上より,個人的選好の自由,社会の決定性,無関係選択肢からの独立性を満たす任意の厚生関数は,ある個人を独裁者にしてしまいます.よって,個人的選好の自由,社会の決定性,無関係選択肢からの独立性,非独裁制を満たす厚生関数は存在しません.
\end{proof}

確かにアローの不可能性定理は示されました.

個人的には独裁者の卵が1つ1つの選択肢に自分の決定力をパタパタパタと波及させながら独裁者へと成長していくダイナミクスに不覚の涙を禁じえませんが,みなさんはどうだったでしょうか.

