\section{民主的な決定は難しい}
我々はしばしば,「みんなで何かを決める」という状況に直面することがあります.例えば,友人と一緒にいて「今日の昼ご飯,何を食べようか?」などとなることがあるかもしれません.今仮に,Lupaくん,Maxeくん,Nymwaくんの3人が昼ご飯に何を食べに行くか相談しているとしましょう.それぞれの人は以下のように主張しました.
\begin{align*}
    Lupa &「私は牛丼が食べたい.その次にカレー.ラーメンは別にいいや.」\\
    Maxe &「僕はカレーが食べたい.別にラーメンでもいいけど.牛丼はいいや.」\\
    Nymwa &「ラーメン自明.牛丼は許す.カレー?アリエナイ…」
\end{align*}
みんなバラバラの意見になってしまいました.このままでは昼ご飯が食べられないので,Nymwaくんが残りの2人にこう聞きました.
\begin{align*}
    Nymwa &「みんな,牛丼とカレーならどっちがいい?」
\end{align*}
LupaくんとNymwaくんは,カレーより牛丼を食べたがっています.一方Maxeくんはカレーが食べたい.多数決を取ったところ,2対1でカレーよりは牛丼ということになりました.Nymwaくんはさらに聞きます.
\begin{align*}
    Nymwa &「じゃあ,牛丼とラーメンならどっちがいい?」
\end{align*}
MaxeくんとNymwaくんは,牛丼よりラーメンが食べたいです.Lupaくんは逆なので,これもまた2対1で牛丼よりラーメンということになります.
\begin{align*}
    Nymwa &「1位ラーメン,2位牛丼,3位カレー,ラーメン食べよう.」
\end{align*}
Nymwaくんはラーメンが食べられると思って非常に喜びました.しかし,ここでLupaくんが言いました.
\begin{align*}
    Lupa &「ラーメンとカレーならどっちがいい?」
\end{align*}
すると,2対1で「カレーはラーメンよりも良い」ということになってしまいます.3位のカレーは1位のラーメンより良い.あれ?

このように選択肢の取りうるペアすべてに対して多数決を取った場合,選択肢の間で「ラーメン > 牛丼 > カレー > ラーメン$\dots\dots$」と堂々巡りが起こることがあります.これを「投票のパラドックス」とか「コンドルセのパラドックス」と言います.

Lupaくんの提案によって,このようなパラドックスが生じてしまったわけなのですが,もしLupaくんがNymwaくんのラーメン推しに何も文句を付けなかった場合,実際に3人はラーメンを食べに行っていたかもしれません.しかし,その場合3人中2人は「ラーメンよりはカレーの方が良い」のであって,「カレーではなくラーメンを選ぶ」というこの3人の集団的な決定に対しては,意思を同じくする個人がNymwaくんただ一人になってしまいます.つまり,Nymwaくんはこの集団の意思決定において,「他の誰がなんと言おうがラーメンは他の選択肢よりも良い」と決めさせる「独裁者」となっているとも言えます.果たしてこれは「民主的な決定」と言えるのでしょうか.

このような決め方の手法には,どうも多数のパラドックスが見つかっているそうです.では,民主的と言えて,かつ堂々巡りの起きないような決め方の方式というのは存在するのでしょうか.

