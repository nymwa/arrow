\section{アロー4条件は妥当か?}
この章ではアロー4条件についてもうすこし深掘りしたいです.

その前に,実は今まで厚生関数と言ってきたものなんですが,世間では社会的厚生関数と呼ばれているので,以後はそのように呼びます.今まで厚生関数と呼んでいたのは,得体のしれない何か名前の長いものという印象を持たれたくないということに依ります.アローの不可能性定理の証明はとても難しいと思うのですが,名前を理由にして読むのを諦められたくなかったからそうしました.許してください.

アローの不可能性定理によって,「民主制に不可欠」なアロー4条件を兼ね揃えた社会的厚生関数は存在しないことがわかりました.一見,「民主制は必ずヒトラーのような独裁者を産むのだ!!」だとか「人々に自由にやらせるといずれどこかで綻びが生じる.自由や権利は制限されるべきだ!!」というようなことを主張しているかのようにも思えるかもしれませんが,それは正しいとは言えません.というのも,これは個人的な価値観も入っているのかもしれませんが,民主制というのは単に個人的選好を社会に反映させるだけのものではなく,個人間で意見が対立していたりバラバラだったりして社会的に何を選択すれば良いかわからないような時は,議論を重ねて意思決定をしていくものだからです.4条件を満たす社会的厚生関数が存在しないことと現実に独裁制が生まれることとは関係はないですし,民主制を否定するために持ち出すのも無理があります.ただしかし,完璧な民主的投票制度は存在しないというのは言えるでしょう.

また,個人の選好だけをもとにして社会的な意思決定をするというのは,危険で薄気味悪いもののようにも思えないでしょうか?なんの議論もせずに,「社会の殆どの人が戦争を望んでいる.だから我が政府は今すぐ戦争を始めるべきだ」とするような決定が正しいと言えるでしょうか?この時,「我々の決定は公正な手続きに基づいている」として人々を沈黙させてしまってもよいでしょうか?

すこしSFっぽいですが,もしも誰もが人間らしい感情を持たなくなり,皆が個人の利益でなく社会の利益のみを追求する滅私社会になったら,個人の選好だけで全会一致で意思決定できてしまうことでしょう.自我を失い誰もが規則正しく品行方正に暮らすそのような社会を合理的だとみなして肯定することができますか\footnote{肯定できる派の方は『盗まれた街』でも読むといいんじゃないですかね.}?個人的選好は多様で,社会的選好の決定はどうしても困難にならざるを得ません.一方,アローの不可能性定理からわかるようにどうしてもすべての個人的選好を反映させることはできません.個人の意見はある程度無視されざるを得ないとも言えます.こうなると,個人的選好だけから社会的な判断をしていいのかという問いは,人間とは何か\footnote{人間の個人的選好が多様であることなど.},倫理とは何か\footnote{功利主義的な立場に立ってすべてを決めていいのかなど.},正義とは何か\footnote{少数者の意見を黙殺していいのかなど.}というような問題にまでなってきてしまうようにも思えますが,そもそも議論や交渉を伴わずに個人的選好だけから社会なレベルのことを決めることの限界というのもあるのだろう,と個人的には思います.

しかし,やはりアローによってこれが発表された当時,その分野の人たちにとってこの定理はかなり衝撃的なことだったらしく,解決策を色々な人が考えたみたいです.20世紀後半に色んな偉い人たちが考えた知見を参考にして,この章ではそもそもアロー4条件や社会的選好が満たすべき性質が民主制に不可欠なのかということを主眼に議論していきたいと思います.

\subsection{選択集合と社会的決定関数}
アローの不可能性定理に対して,経済学者センは社会的厚生関数の条件「社会的選好は弱順序」という条件を緩めることでの解決を考えました.このような関数を社会的決定関数といい,社会的選好が弱順序でなく,「反射性,完全性,非循環性」を満たすものであればアロー4条件を満たす社会的決定関数が存在することを示しました.

非循環性とは,「$x_1 \succ x_2 \succ x_3 \succ \dots \succ x_{n-1} \succ x_n \to  x_1 \succeq x_n$」という性質です.社会的決定関数は,社会的厚生関数の値域の社会的選好に対する制限である弱順序の推移性を非循環性まで弱めたものだと言えます.今まで社会的選好$\succeq$にも個人的選好と同じ強さの制限を課してきたわけですが,それを少し変えて緩めてみましょうということです.以降は$\succeq$は推移性を満たすことを前提としません.図12に反射性,完全性,非循環性を持つ選好の例を示します.

\begin{figure}[!h]
    \centering
    {\scalefont{0.9}
        \begin{tikzpicture}
            \node (1) at (1,2) {桃};
            \node (2) at (0,0) {栗};
            \node (3) at (2,0) {柿};
            \draw [thick, <->](1) -> (2);
            \draw [thick, <-] (1) -> (3);
            \draw [thick, ->] (2) -> (3);
        \end{tikzpicture}
        \begin{tikzpicture}
            \node (1) at (0,2) {桃};
            \node (2) at (2,2) {栗};
            \node (3) at (0,0) {柿};
            \node (4) at (2,0) {梅};
            \draw [thick, <-] (1) -> (2);
            \draw [thick, <->](1) -> (3);
            \draw [thick, <-] (1) -> (4);
            \draw [thick, <-] (2) -> (3);
            \draw [thick, <->](2) -> (4);
            \draw [thick, ->] (3) -> (4);
        \end{tikzpicture}}
        \caption{反射性,完全性,非循環性}
\end{figure}

図12はいずれも推移性を満たしていません.社会的厚生関数の場合ではこのような場合は扱えませんが,仮に社会が図12の左の例のような選好をしている場合,社会的には桃を選択するとしてもよいかもしれません.

図12の右の例で,例えば何か1つしか食べられないとしたら桃を選べば良さそうです.逆に何か1つは食べられないとしたら,柿を食べなければ良いでしょう.桃以外から選べと言われたら栗と梅になります.ところで,推移性を非循環性まで弱めた時いつもこのように$M$の部分集合の中で最良と決められる要素が存在するのでしょうか.それについて見ていこうと思います.

\begin{dfn}[選択集合]
    $\forall y \in M (x \succeq y)$であるような選択肢$x$を最良要素と言います.最良要素の集合を選択集合と言います.
\end{dfn}

最良要素は図12で言うと他のすべての要素から矢印が向いてるものです.

\begin{thm}
    $\succeq$が反射性,完全性を満たす時,$M$の空でない任意の部分集合に対して選択集合が空集合でないための必要十分条件は,$\succeq$が非循環性を持つことです.
\end{thm}

図12のようなグラフでどんな空でない部分集合に対しても最良要素が存在するための必要十分条件です.証明しましょう.

\begin{proof}
(必要性)非循環性を満たさないと仮定すると,ある$m$個の選択肢$x_1,x_2,\dots,x_m$に対して,$x_1 \succ x_2 \succ \dots \succ x_m \succ x_1$となります.しかし,この$m$個の選択肢の集合には最良要素が存在しません.この対偶を取ると必要条件となります.

(十分性)非循環性が成り立つことを仮定します.完全性より,任意の選択肢のペア$x,y$に関して,$x\simeq y$または$x \prec y \lor x \succ y$です.

今,$M$の任意の空でない部分集合$V$について考えます.$V$の任意の選択肢のペア$x,y$に対して,$x \simeq y$の時,すべての要素が最良要素なので選択集合が存在します.非循環性も満たしています.

次に$V$のある2選択肢$x,y$で$x \succ y$となっている時を考えます.集合$\{x,y\}$では$x$が最良要素です.今$V$のすべての選択肢$z$に対して$x \succeq z$となるなら,$x$は$V$の最良要素となります.ある選択肢$z$で$z \succ x$ならば,非循環性より$z \succeq y$なので,集合$\{x,y,z\}$で$z$が最良要素となります.このように最良要素となり得る選択肢を集合に加えていき最良要素の候補を更新していけば,どこかで1つの最良要素が確定し,結局選択集合が存在します,
\end{proof}

社会的選好の推移性を非循環性まで弱めても,任意の選択肢の部分集合から同率含めて1位のものが選べるということがわかりました.社会的決定関数による社会的選好から何かを選択したいときは選択集合の要素を選べば良いということになります.

アロー4条件を満たす社会的決定関数の例には例えば以下のようなものがあります.
\begin{equation}
    x \succeq y \leftrightarrow \forall i \in M (x \doteq_i y) \lor \exists i \in M (x >_i y) \tag{*}
\end{equation}
これは,「選択肢$x$と$y$に対して,みんなが$x$と$y$は同じぐらい良いと思ってるか,誰か1人でも$x$は$y$より良いと思っていれば,社会的に$x$は$y$と少なくとも同程度以上には良い」ということです.

この社会的決定関数がアロー4条件を満たすことを確認しましょう.

一番わかりやすいのは社会の決定性でしょう.すべての個人が$x > y$と思っている時,$x \succeq y$ですが,$x \not \preceq y$なので,社会的にも$x \succ y$です.無関係選択肢からの独立性も,(*)式より$x,y$の社会的選好は$x,y$の個人的選好のみから決まることから明らかです.非独裁制も,全個人の社会的選好への影響力が等しいことから明らかでしょう.

個人的選好の自由は少し難しいです.可能なすべての個人的選好に対して定まる社会的選好が反射性・完全性・非循環性を満たすかを調べれば良いです.個人的選好は弱順序であることに気をつけておいてください.

$x \succeq x \leftrightarrow \forall i (x \doteq_i x) \lor \exists i (x >_i x)$
ですが,$\forall i (x \doteq_i x)$が真なので,反射性を満たします.

完全性が成り立たないと仮定します.すると,ある異なる2選択肢$x,y$で$x \not \succeq y \land x \not \preceq y$です.すると,
\begin{align*}
    &x \not \succeq y \land x \not \preceq y \\
    \to & \sim (\forall i (x \doteq_i y) \lor \exists i (x >_i y)) \land \sim (\forall i (y \doteq_i x) \lor \exists i (y >_i x)) \\
    \to & (\exists i (x \not \doteq_i y) \land \forall i (x \leq_i y)) \land (\exists i (y \not \doteq_i x) \land \forall i (y \leq_i x))\\
    \to & \exists i (x \not \doteq_i y) \land \forall i (x \doteq_i y)\\
    \to & \exists i (x \not \doteq_i y) \land \sim \exists i (x \not \doteq_i y)
\end{align*}
と矛盾します.よって完全性も満たします.

非循環性は$x_1 \succ x_2 \succ \dots \succ x_n \to x_1 \succeq x_n$です.前件を仮定します.前件は,
\begin{equation*}
    \left\{
    \begin{array}[h]{l}
        x_1 \succeq x_2 \land \dots \land x_{n-1} \succeq x_n \\
        x_2 \not \succeq x_1 \land \dots \land x_n \not \succeq x_{n-1} 
    \end{array}
    \right.
\end{equation*}
ということなので,
\begin{equation*}
    \left\{
    \begin{array}[h]{clc}
        \forall i (x_1 \doteq_i x_2) \lor \exists i (x_1 >_i x_2) && (a_1)\\
        \vdots &\hspace{2em}& \vdots \\
        \forall i (x_{n-1} \doteq_i x_n) \lor \exists i (x_{n-1} >_i x_n) && (a_{n-1})\\
        \exists i (x_2 \not \doteq_i x_1) \land \forall i (x_2 \leq_i x_1) && (b_1) \\
        \vdots && \vdots \\
        \exists i (x_n \not \doteq_i x_{n-1}) \land \forall i (x_n \leq_i x_{n-1}) && (b_{n-1})
    \end{array}
    \right.
\end{equation*}
です.

$(b_1)$より,$\exists i (x_2 \not \doteq_i x_1)$ですが,二重否定の導入により,$\sim \forall i (x_1 \doteq_i x_2)$.

これと,$(a_1)$より,選言三段論法によって$\exists i (x_1 >_i x_2)\hdots (c)$.

$(b_k)$より,$\forall i (x_{k+1} \leq_i x_k)$.よって$\forall i (x_2 \geq_i \dots \geq_i x_n)$.すなわち,$\forall i (x_2 \geq_i x_n)$.これと式(c)より,$\exists i (x_1 >_i x_n)$なので,$\forall i (x_1 \doteq_i x_n) \lor \exists i (x_1 >_i x_n)$.よって非循環性も満たします.

以上より,(*)式はその定義域に可能などのような弱順序の個人的選好の組があっても,反射性・完全性・非循環性を満たす社会的選好をただ1つ定めるということがわかりました.

こうすれば,「民主制に不可欠」とされる4条件のすべてを満たすのですが,ところで最初に出てきた昼ご飯の例ではどうでしょう.どの2つの選択肢に対しても,少なくとも1人がどちらかを良い,例えば「カレーは牛丼より良い」などと主張しているので,(*)の社会的決定関数を用いると,「カレー $=$ 牛丼 $=$ ラーメン」という結果になってしまいます.言われてみれば確かにそうで,あの3人の集団ではこの3選択肢は等価値です.この場合,あとは3人で話し合って決めてくださいという感じですが,このように見ると,この社会的決定関数,選択肢$x$と$y$で,誰か1人でも抵抗する者がいれば,社会的に$x$か$y$か決められないという,非常に優柔不断な関数だとわかります.

アロー4条件を満たそうとすると,社会の全員を納得させない限り何も決められないということになっては困ってしまいます.すると,「民主制に不可欠」などと呼んでいる条件自体に問題があったのではないかという疑念がますます深まります.具体的に少し見ていきたいと思います.

