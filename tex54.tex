\subsection{社会の決定性と自由主義のパラドックス}

社会の決定性は明らかに受け入れられることのように思われます.全員が賛成しているのだったなら,当然社会的にも認められるべきだというのは,尤もらしいという感じがします.

しかし,社会の決定性を認めると,奇妙なことが起こり得るのです.

\begin{dfn}[自由主義]
    任意の個人$i$は少なくとも1つの異なる選択肢のペア$x,y$に対して決定力を持つ.
\end{dfn}

つまり,自由主義が認められる社会においては,どの個人もある選択肢の組に対して$D_i(x \succ y) \land D_i(x \prec y)$ということです.今後,このことを個人$i$は,$x,y$の社会的選好を決める権利があると言います.

\begin{dfn}[弱い自由主義]
    少なくとも2人のある個人$j,k$は,ある2つの異なる選択肢のペア$x,y$と$z,w$に対してそれぞれ決定力を持つ.
\end{dfn}

これは明らかに自由主義を弱めたものです.

さて,「これが自由主義だ」と言われてもなんのことなのかよくわかりません.2つ例を挙げて説明したいと思います.

\subsubsection*{例1.父と息子の会話(センのパラドックス改)}
ある家庭で父と息子が話をしてるところを想定してください.
\begin{align*}
    息子 & 「お父さん,僕○○大学に行きたいんだよ」 \\
    父   & 「うちは貧乏だから学費は出せないよ」 \\
    息子 & 「でも行きたいんだよ」 \\
    父   & 「大学なんて行ってないで就職しなさい」 \\
    息子 & 「まだ就職はしたくないんだよ」 \\
    父   & 「なんだと?無職になる気か?そんな子に育てた覚えはないぞ」 \\
    息子 & 「でも,就職するぐらいだったら無職の方がマシだね」 \\
    父   & 「なんだと!?それだけは絶対に許さんぞ!!」
\end{align*}

さて,今自由主義と社会の決定性が成立することを仮定しましょう.そして3つの選択肢$x,y,z$を,
\begin{align*}
    就職 : & 息子は就職する \\
    進学 : & 息子は進学する \\
    無職 : & 息子は無職になる
\end{align*}
とします.すると,息子と父の選好は以下のようになります.
\begin{align*}
    父   :& 就職 > 進学 > 無職 \\
    息子 :& 進学 > 無職 > 就職 
\end{align*}
ここで,自由主義の要請について考えます.今,父は息子に学費を出さない自由を持っています.よって,社会的(というか家庭的)にも「$就職 \succ 進学$」が選好されます.また,息子には就職するか無職になるか選択する自由があるはずです.よって,社会的にも「$無職 \succ 就職$」が選好されます.また,父も息子も$進学 > 無職$では共通しているので,社会の決定性より「$進学 \succ 無職$」です.しかし,これでは図17のように循環が起きてしまいます.

ここで,今まで同様「社会的」という言葉を用いてきました.父が学費を出すか,息子が働くか,というのは家庭の問題だというのは,おそらく正しいでしょう.社会的選好や個人の自由というのがこの例に関係あるのかという感じもしますが,ここで言う自由主義は,父・息子が持つ自分の社会的に容認された権利の中での選択は,社会的にも容認されるべきだという立場を取っていることに気をつけてください.

例えば自分が東京に住むか,大阪に住むかを決めるのは自分の権利であり,そのような問題を社会的決定の対象に含めていいのかというのは確かに問題です.しかし,今まで個人的選好をもとに決定を下すことの限界や問題を見てきたわけで,そのような決定の方法を見限らずに,どこまでできるのか,どう理論を構築すればいいのかを考えることは意味があることなのだろうと思います.

\begin{figure}[!h]
    \label{fig:17}
    \centering
    \begin{tikzpicture}
        \node (1) at (1, 1.7) {就職};
        \node (2) at (0, 0) {進学};
        \node (3) at (2, 0) {無職};
        \draw[thick,->] (1) -> (3);
        \draw[thick,->] (3) -> (2);
        \draw[thick,->] (2) -> (1);
    \end{tikzpicture}
    \caption{社会的選好}
\end{figure}

\subsubsection*{例2.新築の屋根の色(ギバードのパラドックス)}
AさんとBさんが二軒並んだ新築の一軒家を購入しました.AさんとBさんは塗装業者に行って屋根を塗ってもらうことにしました.さて,今Aさんは「Bと同じ色は嫌だなあ」と思っていて,Bさんは「Aと同じ色がいいなあ」と思っています.

\begin{align*}
    業者 & 「赤色と緑色ならお安くできます」 \\
    Aさん& 「じゃあそうしよう」\\
    Bさん& 「そうしようそうしよう」\\
    Aさん& 「じゃあ,僕は赤色にしようかな」\\
    Bさん& 「じゃ,私も赤色で」\\
    Aさん& 「え?じゃあ緑色に変えます」\\
    Bさん& 「えー.やっぱ緑色で」\\
    Aさん& 「うーむ.赤色にしますわ……」\\
         &  (以下略)
\end{align*}

と,このように循環が起こってしまいます.なぜこのようなことが起こったのか理由は明らかかと思いますが,一応説明します.まず,社会的な選択肢は,Aの選択する色XとBのそれYを$(X,Y)$とすると,以下の4つ
\begin{equation*}
    (赤,赤),(赤,緑),(緑,赤),(緑,緑)
\end{equation*}
になります.この時,A,Bの個人的選好は,
\begin{align*}
    A : & (赤,緑) \doteq_A (緑,赤) >_A (赤,赤) \doteq_A (緑,緑) \\
    B : & (赤,赤) \doteq_B (緑,緑) >_B (赤,緑) \doteq_B (緑,赤)
\end{align*}
です.そして,Aは$(赤,緑) \succ (緑,緑)$,$(緑,赤) \succ (赤,赤)$とする自由があり,Bは$(赤,赤) \succ (赤,緑)$,$(緑,緑) \succ (緑,赤)$とする自由があります.そのため,社会的選好は図18のようになり,循環してしまいます.

\begin{figure}[!h]
    \centering
    \begin{tikzpicture}
        \node (1) at (0,3) {(赤,赤)};
        \node (2) at (4,3) {(緑,赤)};
        \node (3) at (4,0) {(緑,緑)};
        \node (4) at (0,0) {(赤,緑)};
        \draw[thick,->] (1) -- (2);
        \draw[thick,->] (2) -- (3);
        \draw[thick,->] (3) -- (4);
        \draw[thick,->] (4) -- (1);
        \node (5) at (2,3.3) {Aの自由};
        \node (6) at (5,1.5) {Bの自由};
        \node (7) at (2,-0.4) {Aの自由};
        \node (8) at (-1,1.5) {Bの自由};
    \end{tikzpicture}
    \caption{社会的選好}
\end{figure}

この例では社会の決定性は全く関係しません.社会の決定性だけが自由主義の下で循環を起こす要因ではないということです.しかし,筆者の力不足と調査不足のため,このような場合への考察はここではしません.誰か教えてください.

さて,例1,2のように,個人の自由というものを社会に容認させると,社会的選好の循環が起こりえます.これを自由主義のパラドックス,あるいはリベラル・パラドックスと言います.

\begin{thm}[センの不可能性定理]
    弱い自由主義,個人的選好の自由,社会の決定性を同時に満たす社会的決定関数は存在しません.
\end{thm}

\begin{proof}
個人$j,k$,選択肢$x,y,z,w$で,$j$は$x,y$の社会的選好を決める権利があり,$k$は$z,w$の社会的選好を決める権利があるとします.

今,この2人に決める権利のある選択肢に共通の要素があるとします.2つの要素が共通する時はお互い相反する選好をしうるので,明らかに個人的選好の自由を満たしません.1つの要素が共通するときについて考えます.例えば$x = z$とします.

個人的選好の自由より,どのような個人の選好に対しても社会的選好が1つ定まります.そこで,$x >_j y$,$w > _k x$,$\forall i (y >_i w)$と仮定します.この時,$j$の権利より,$x \succ y$.また,$k$の権利より$w \succ x$.そして,社会の決定性より$y \succ w$です.しかし,これは$x \succ y \succ w$となり,循環を生みます.このため,非循環性を満たしません.

次に,共通する選択肢がないときを考えます.$x >_j y$,$z >_k w$,$\forall i (w >_i \land y >_i z)$と仮定します.この時,$j$の権利より$x \succ y$.$k$の権利より$z \succ w$.社会の決定性より$w \succ x,y \succ z$.よって,$x \succ y \succ z \succ w \succ x$となり,非循環性を満たしません.
\end{proof}

さて,ここまで見てきた通り,今まで社会的選好が依るべきとした条件である社会の決定性は,自由主義と相反するものだということがわかりました.そこで,自由主義か社会の決定性のどちらかを制限することを考えたいです.しかし,自由主義を制限することを考えると,「具体的にどの個人のどの権利を制限するか」という問題になってしまいます.まずは社会の決定性がいかに自由主義と矛盾を生じるか考えます.

