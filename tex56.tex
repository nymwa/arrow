\subsection{開かれた社会的決定}

今,客がウェイターに注文を取っているところを想定してください.以下のように会話しているとします.
\begin{align*}
    ウェイター &  「本日はトンカツとステーキのみとなっております」 \\
    客         &  「じゃあ,ステーキにするよ」 \\
    ウェイター &  「あっ.伊勢海老のグリルもお出しできますが」 \\
    客         &  「おっと.えー,うーん.じゃあ,やっぱりトンカツにするよ」 
\end{align*}
さて,今このお客さんにとって集合$\{トンカツ,ステーキ\}$の最良要素はステーキです.しかし,集合$\{トンカツ,ステーキ,伊勢海老のグリル\}$の最良要素はトンカツとなってしまいます.伊勢海老のグリルという新しい選択肢が増えたことで,すでに決めた選好が変わっているのですが,そのような変化は一見不自然なように思えないでしょうか\footnote{もちろん,客が全く合理性に欠く判断をしたとは断定できませんが.この客の心理を合理的に説明する方法を考えてみるのも面白いかもしれません.}.

無関係選択肢からの独立性が要求していたのは,選択肢の集合の任意の部分集合での社会的選好が変化してはならないという条件でした.上の例のようなことが社会的に起こってはならないということです.ここで,無関係選択肢からの独立性を厳しくした条件である選択肢の平等性を仮定した時に社会的厚生関数が引き起こす奇妙な事態について説明したいと思います.

\subsubsection*{ハンソンの定理}
\begin{thm}[ハンソンの定理]
    個人的選好の自由,個人の平等性,選択肢の平等性を満たす社会的構成関数は,任意の選択肢を同順位としなければなりません.
\end{thm}

誰がなんと言おうと,すべての選択肢が同率1位となってしまいます.そんなこと,あるんでしょうかという感じです.

\begin{proof}
個人の平等性と選択肢の平等性を満たす社会的厚生関数で,2選択肢$x,y$に対して図21のような個人的選好を考えます.このとき,個人の数は偶数人であることを仮定します.社会全体は2n人とし,その内n人を個人の集合A,残りのn人を集合Bとします.
\begin{figure}[!h]
    \centering
    \begin{tabular}[!h]{|c|c|c|c|} \hline
         & A & B & 社会 \\ \hline
        x,y & $x > y$ & $x < y$ & ? \\ \hline
    \end{tabular}
    \caption{個人的選好} 
\end{figure}
さて,今個人の平等性が満たされているので,集合Aの人と集合Bの人をすべて入れ替えても社会的選好は変わりません.つまり,図22のようになります.
\begin{figure}[!h]
    \centering
    \begin{tabular}[!h]{|c|c|c|c|} \hline
         & A & B & 社会 \\ \hline
        x,y & $x < y$ & $x > y$ & ? \\ \hline
    \end{tabular}
    \caption{個人的選好} 
\end{figure}
さらに,選択肢の平等性より,選択肢$x,y$を入れ替えても結果は同じです.つまり,図23のようになります.
\begin{figure}[!h]
    \centering
    \begin{tabular}[!h]{|c|c|c|c|} \hline
         & A & B & 社会 \\ \hline
        y,x & $y > x$ & $y < x$ & (?のx,yを入れ替えたもの) \\ \hline
    \end{tabular}
    \caption{個人的選好} 
\end{figure}

もし,?が$x \succ y$だと仮定すると,図22と図23で同じ個人的選好から異なる社会的選好が定まることになり矛盾します.$x \prec y$も同様です.社会的厚生関数の社会的選好は完全性を満たしているので,$x \simeq y$ということになります.

以上より,個人・選択肢の平等性が満たされ,偶数人で意見が半数ずつ対立すれば,社会的選好は2選択肢を等価とすることがわかります.これは別に奇妙でも何でもなく,普通に受け入れられる事実です.

さて,次に3選択肢$x,y,z$に関して,個人的選好が図24のようになっているとします.
\begin{figure}[!h]
    \centering
    \begin{tabular}[!h]{|c|c|c|c|} \hline
         & A & B & 社会 \\ \hline
        x,y & $x > y$ & $x < y$ & $x \simeq y$ \\ \hline
        y,z & $y < z$ & $y > z$ & $y \simeq z$ \\ \hline
        z,x & $z > x$ & $z > x$ & $z \simeq x$ \\ \hline
    \end{tabular}
    \caption{個人的選好}
\end{figure}
$x,y$,$y,z$間では半数ずつ対立しているので社会的選好はそれらを等価とします.さらに社会的選好の推移性より$z \simeq x$です.選択肢の平等性より,無関係選択肢からの独立性が言えるので,結局,2選択肢間で全員が同じ選好をした時は社会的にその2選択肢は等価とされることがわかりました.

そうすると,結局図25のような場合を考えれば,推移性によって選択肢が同順位であることが波及し,どのような個人的選好に対しても社会的に等価としなければならないことがわかります.
\begin{figure}[!h]
    \centering
    \begin{tabular}[!h]{|c|c|c|c|} \hline
         & A & B & 社会 \\ \hline
        x,y &    ?    &    ?    & $x \simeq y$ \\ \hline
        y,z & $y > z$ & $y > z$ & $y \simeq z$ \\ \hline
        z,x & $z > x$ & $z > x$ & $z \simeq x$ \\ \hline
    \end{tabular}
    \caption{個人的選好}
\end{figure}

次に社会が奇数人である時を考えます.図26のような場合を考えます.Aがn人,Bが{\bf n-1}人とします.
\begin{figure}[!h]
    \centering
    \begin{tabular}[!h]{|c|c|c|c|c|c|} \hline
         & i & j & A & B & 社会 \\ \hline
        x,y & $x > y$ & $x < y$ & $x > y$ & $x < y$ & $?_0$ \\ \hline
        y,z & $y < z$ & $y > z$ & $y > z$ & $y < z$ & $?_1$ \\ \hline
        z,x & $z > x$ & $z < x$ & $z < x$ & $z > x$ & $?_2$ \\ \hline
    \end{tabular}
    \caption{個人的選好}
\end{figure}
個人の平等性より,$?_0$,$?_1$,$?_2$は同じとなります.しかし,$?_0$が$x \succ y$の時,$?_1$が$y \succ z$となり,推移性より$?_2$は$z \prec x$となり矛盾します.$?_0$は$x \prec y$ともなりえず,$x \simeq y$です.同様にして$y \simeq z$,$z \simeq x$です.ここから,n人とn+1人が対立する場合,社会的に等価とみなすことになります.

次に,Aがn+1人,Bがn人であるとします.図27のような場合を考えます.
\begin{figure}[!h]
    \centering
    \begin{tabular}[!h]{|c|c|c|c|c|} \hline
         & A & B & 社会 \\ \hline
        x,y &    ?    &    ?    & $x \simeq y$ \\ \hline
        y,z & $y > z$ & $y < z$ & $y \simeq z$ \\ \hline
        z,x & $z < x$ & $z < x$ & $z \simeq x$ \\ \hline
    \end{tabular}
    \caption{個人的選好}
\end{figure}
n+1人とn人が対立する場合社会的に等価になることと推移性より,x,yに関するどのような個人的選好に対しても,x,yは社会的に等価となります.

以上より,個人と選択肢の平等性を満たす社会的厚生関数は,すべての選択肢を社会的に同順位としなければなりません.
\end{proof}

公正な投票方式として,個人の平等性,選択肢の平等性,社会的選好の推移性は当然満たされるべきだと考えるのは自然なことに思えます.しかし,どのような投票方式も個人の平等性,選択肢の平等性,社会的な推移性を要求するとすべての選択肢を同順位とせざるを得ないという結果が導かれてしまいました.

民主的な社会では当然個人の平等性は満たされているべきです.つまり,現実の投票方式では選択肢の平等性か推移性のどちらかが犠牲になっているということです.しかし,普通我々が行う投票方式は皆社会的な推移性を満たしています\footnote{推移性を満たしていない場合,選挙結果が定まらないような事態が起き得,何も決められないのですが,それでは投票を行う意味がないです.}.つまり,我々は選挙方式の中で選択肢の平等性を完全に満たすことはできません.

例えば,我々が国政選挙などで行っている単記投票方式では投票結果が推移性を満たしています.というのも,投票結果に循環は起こりえないからです.よって,選択肢の平等性を満たしていないということになってしまいます.

これは,単記投票では戦略的投票を許してしまうということです.無関係選択肢からの独立性を満たさないと戦略的投票が起きうるということをボルダ得点法の説明の時に述べました.選択肢の平等性を満たしていないからと言って,必ずしも戦略的投票が起きるとは言えませんが,単記投票では起きてしまいます.例えば,選択肢A,B,Cから単記投票で選ぶ時,「AさんとBさんは大きな政党から推薦をもらっているけど,Cさんはもらっていないから当選の見込みはないだろう.本当は考えとしてはCさんが自分に最も近いが,Aさんには当選してほしくないので,やむを得ずBさんに投票しよう」というような投票行動は自然なものとなってしまいます.

5.3節で選択肢の平等性が投票の信頼性を保証するという旨のことを言いました.しかし,おそらく我々は国政選挙などで各党の政策マニフェストを見て投票する時に,マニフェストの党名を隠して読むか,隠さず読んで判断するかで結果が変わってくるのではないかと思います\footnote{比例代表制の選挙の前日に,例えば突然自民党と共産党の名前が入れ替わったらもともと共産党だった方の政党が勝つと思いませんか?}.5.3節で述べたのはこの条件の「選択肢の名前にかかわらず平等」という点のみだけで,「選択肢の名前を入れ替えても平等」という点のみまで保証することは少し厳しそうです.

\subsubsection*{開かれた社会的決定}
ハンソンの定理から推移性を満たす投票方式は選択肢の平等性を満たしていないことがわかります.アロー4条件から無関係選択肢からの独立性を除けば選択肢の平等性を満たしていないことがことさらに問題にはなりません.しかし,無関係選択肢からの独立性を満たしていないということは,2つの選択肢に関する社会的決定を下すためには,その2つの選択肢以外の選択肢の間の選好関係についても知っていなければならないということを意味します.極端な言い方をすれば,「りんご」と「みかん」の選好を決めるときに,「90式戦車」と「10式戦車」の選好についても考慮しなければならないということです.さすがにそんな馬鹿げたことが問題となる投票はないだろうとは言えるのでしょうが,決定というのは常に暫定的なもので,新たな選択肢が追加されたり,既存の選択肢が削除されたりした時に,すでに決めていた選好が変わることは認めてもいいのかもしれません.一度「りんごはみかんより良い」と決めたらそれが永久に覆らないというのは,むしろ不自然にも思えます.

そこで,「開かれた社会的決定」というものを考えます.

\begin{dfn}
    開かれた社会的決定とは,すでに暫定的に限定された選択肢の集合に,新しい選択肢が追加されたり,あるいはすでにある選択肢が除外されたりした時に,すでに決まっている個人的選好に全く変化がないとしても,すでに決まっているどのような社会的選好も変更されうる可能性があることをすべての個人が了承した上で行われる社会的決定のことです.
\end{dfn}

既存の選択肢だったA党が選挙直前に分裂してB党とC党になったりすると,もともとA党が勝ちそうだったのに,B党もC党も選挙で負けてしまうというように選択肢の消失や追加が社会的選好に影響を与えることをみなさんちゃんと理解してどのように投票するか決めてくださいということです.

それがどうした,という感じもあります.しかし,ハンソンの定理から社会的選好が推移性を満たすなら選択肢の平等性は保証されず,社会的決定は開かれたものであるべきだということが言えます.ハンソンの定理によって,開かれていることを社会に課すことの是非を検討する道理が生まれるのです.

開かれた社会であるべきという態度は言い換えると,単記投票のように個人の細かい選好が見えにくい投票方式では,無関係選択肢からの影響が見えにくいため,各個人が各選択肢をどのような順位で選好するかということも社会的決定に生かそうというものです.これはボルダ得点法を正当化するものです

ボルダ得点法の欠陥はすでに見たとおりです.結局ボルダ得点法は良いのか悪いのかどっちなんだという感じですが,ここで言いたいのは,{\bf 別に無関係選択肢からの独立性を満たしてないからと言って,そんなに深刻な問題ではないですよ}という,3章の伏線の回収と,{\bf 多数決はどうあがいても推移性か平等性のどちらかを失う不完全なものなのですよ}ということです.ボルダ得点法が良いか悪いかはよくわかりません\footnote{利点,欠点というような個性はそれぞれありますが.}.

