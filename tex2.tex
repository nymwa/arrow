\section{準備}
集団がいくつかの選択肢に順位付け・ランク付け(同率を許す)をするという状況を考えてみましょう.今仮に,ある国で今年の人気歌手ランキングを作ることになったとします.さて,どのように作成すればいいでしょうか.

もしもその国のすべての人の各歌手への好感度を測定することができるとしたら,好感度の和が高い順に順位を付けていけば良いのかもしれません.しかし,「○○さんは10好きだけど,××さんは3好きだなぁー.」などと,個人の好き嫌いの度合いを量的な尺度で扱うのは無理がありそうです.

そこで,一人ひとりに「あなたの好きな歌手ランキング」を書いてもらって,集計して最終的な国全体のランキングを作成することにしました.しかし,そのようにして集められた結果から一体どのようにして,それぞれの歌手の間に順位付けをすればいいのでしょうか.というか,そもそも順位ってなんなんでしょうか.

まずは,議論に必要な言葉や記号を定義していきましょう\footnote{これ以降で出てくる用語には私がかってに命名したものもあるので,広く使われているものとは限らないことに注意してください.}.

\subsection{選択肢と選好}
\begin{dfn}
    選択肢とは,個人や社会がその価値判断に基づいて評価し,選択したり順序付けたりする対象のことです.
\end{dfn}
\begin{dfn}
    選好とは,ある選択肢が他の選択肢よりも「良い」という判断のことです.個人の選好を個人的選好,その人たちが属する社会の選好を社会的選好と言います.
\end{dfn}

どのような決定方法でも選択肢と,それを選ぶ人は存在します.選択肢は人かもしれませんし,社会の状態かもしれませんし,いろいろです.選ぶ人は自分の判断,すなわち選好に基づいて選択肢を選んだり拒んだりします.

以降,選択肢(Alternatives)全体の集合を$A$,個人をラテン文字の小文字$i,j,k,\dots$で表し,社会に属する全個人の集合を$M$,その部分集合を$V,W,\dots$と表します.$M$はmembersのつもりです.また,選択肢,個人の数は有限とします.

\subsection{記号の定義}
この文書全体で用いられる集合や論理の記号の意味を説明します.なお,命題とは真か偽か判断できる文のことです.
\begin{itemize}
    \item[$\sim$]
        否定.「$\sim P$」は,「Pではない」という命題です.
    \item[$\land$]
        論理積.「$P \land Q$」は,「PとQがともに真である」という命題です.
    \item[$\lor$]
        論理和.「$P \lor Q$」は,「PとQの少なくとも片方は真である」という命題です.
    \item[$\to$]
        含意.「$P \to Q$」は,「PとQはともに真,あるいはPは偽である」,すなわち「PならばQ」という命題です.
    \item[$\leftrightarrow$]
        同値.「$P \leftrightarrow Q$」は,「PとQはともに真,あるいはともに偽」という命題です.
    \item[$\in$]
        属する.「$x \in X$」は,「要素$x$は集合$X$に属する」という意味です.
    \item[$\subset$]
        部分集合,含まれる.「$A \subset B$」は,「集合Aは集合Bに含まれる」という意味です.「$B \supset A$」はこれと同じです.
    \item[$-$]
        差集合.$A-B$で集合AからBの要素を取り除いた集合を意味します.
    \item[$|A|$]
        要素数.$|A|$で集合Aに属する要素数を意味します.
    \item[$A^n$]
        $A$の直積.$A$の要素のnつ組の集合です.$R^2$は平面になります.
    \item[$2^A$]
        べき集合.$2^A$で集合Aのすべての部分集合からなる集合を意味します.
    \item[$\exists$]
        存在量化子.「$\exists x \in X \: P(x)$」は,「集合Xの中にはある要素xが存在して命題P(x)が満たされる」という命題です.    
    \item[$\forall$]
        全称量化子.「$\forall x \in X \: P(x)$」は,「集合Xの中のすべての要素xに対して命題P(x)が成立する」という命題です.
\end{itemize}

論理記号の結合の優先度は$\sim,\exists,\forall$が最も高く,次に$\land,\lor$で,$\to,\leftrightarrow$が最も低いです.$P \to Q \to R$や$P \leftrightarrow Q \leftrightarrow R$は,それぞれ$(P \to Q) \land (Q \to R)$,$(P \leftrightarrow Q) \land (Q \leftrightarrow R)$を意味します.左寄せ・右寄せの規則等は設けません.

\subsection{二項関係}
\begin{dfn}
    集合$X$上の二項関係$R$とは,集合$X^2$上で定義され,任意の$x,y \in X$に対して命題$xRy$が真か偽かを定める関数です.$xRy$が真であることを$xRy$,偽であることを$\sim xRy$または$x\not \! \! R y$と表します.
\end{dfn}

$X = \{グー,チョキ,パー\}$.$X$上の二項関係$xRy$を「xがyに勝つ」とします.このとき,「グー$R$チョキ」,「チョキ$R$パー」,「パー$R$グー」ですが,その他の組み合わせ(負けorあいこ)では例えば「グー$\not \! \! R$パー」です.

選択肢間の選好は,選択肢の集合$A$上の二項関係として表現できます.
\begin{dfn}
    選択肢$x$と$y$に対して,$x \ge y$は,「xはyと少なくとも同じぐらいよい」とする二項関係です.
\end{dfn}
「xはyと少なくとも同じぐらいよい」というのは「xとyが同じぐらい良いか,または明確にxはyよりも良い」という意味であることに気をつけてください.また,$\ge$が意味的に対応していることを確認してください.

\begin{dfn}
    $\ge$の右下に個人または個人の集合を$x \ge_i y$,$x \ge_V y$とつけた場合,それぞれ,「個人$i$は,xはyと少なくとも同じぐらいよいと選好する」,「集合$V$に含まれるすべての個人$j$が,xはyと少なくとも同じぐらいよいと選好する」という二項関係です.
\end{dfn}

\begin{dfn}
    $x \succeq y$は,「社会的にxはyと少なくとも同じぐらいよい」と選好するという二項関係です.
\end{dfn}

「$\succeq$なんて見たことないしぐにゃってしてて怖いよぅ」などとして読むのを止めないでほしいです\footnote{昔の自分はそうでした.}.

$x \ge_M y$と$x \succeq y$が全く異なるものを意味していることに注意してください.$x \ge_M y$は「社会に属するすべての個人が"$x \ge y$"と思っている」ですが,$x \succeq y$は「社会全体では"$x \ge y$"というように決まった」という意味です.これは,例えばある事案で賛否を取ったとき,全ての個人が賛成したのなら社会的にも賛成となるのでしょうが,社会的に賛成だとしても社会の中には反対とした個人がいるかもしれないという違いです.

二項関係$>,<,\doteq,\succ,\prec,\simeq$を以下のように定義します.
\begin{dfn}
    $x>y \leftrightarrow (x \geq y かつ y \not \geq x) \leftrightarrow y<x$
\end{dfn}
\begin{dfn}
    $x \doteq y \leftrightarrow (x \geq y かつ y \geq x)$
\end{dfn}
\begin{dfn}
    $x \succ y \leftrightarrow (x \succeq y かつ y \not \succeq x) \leftrightarrow y \prec x$
\end{dfn}
\begin{dfn}
    $x \simeq y \leftrightarrow (x \succeq y かつ y \succeq x)$
\end{dfn}

$x>y$は「xはyよりも明確に良い」,すなわち,「xをyよりも選好する」という意味になります.

$x\doteq y$は,「xとyは同程度によい」という意味ですが,「xとyはどちらのほうが良いとも言えない」という意味ではないことに注意してください.$x$と$y$が取れる選好は以下の4つにいずれかになり,
\begin{align*}
    1 :& x \geq y かつ y \geq x \hspace{1em} (x \doteq y) \\
    2 :& x \geq y かつ y \not \geq x \hspace{1em} (x > y) \\
    3 :& x \not \geq y かつ y \geq x \hspace{1em} (x < y) \\
    4 :& x \not \geq y かつ y \not \geq x \hspace{1em} 
\end{align*}
「xとyはどちらのほうが良いとも言えない」はこのうち4番目です.ただ,後で出てくる「完全性」の制約のために4番目の場合に注意する必要はあまりないので安心してください.

また,$x\doteq y$は,「$x$と$y$が同一の選択肢である」という意味ではないことにも注意してください.この場合,$x$と$y$は,同じ選択肢かもしれませんし,異なる選択肢かもしれません.$\doteq$が等しいとしているのはその個人が評価する選択肢の価値です.また,$\doteq$の上の点を付けた理由は$=$との混同を防ぐためです.

\subsection{二項関係$\geq$や$\succeq$が満たす性質}
先程定義した$\geq$では,例えば$x \not \geq x$としたとき,「xはxより良いとは言えない」という意味になってしまい,不思議な選好になってしまいます.これから議論する選好がこのようにならないために,満たしてほしい性質をいくつか決める必要があります.
\begin{dfn}[反射性]
    任意の二項関係$R$は,すべての選択肢$x$に対し,$x R x$を満たすとき反射性を持つと言います.
\end{dfn}
\begin{dfn}[完全性]
    任意の二項関係$R$は,任意の{\bf 異なる\footnote{この「異なる」がないと,$xRx$のような場合も含むため反射性の条件を包含してしまいます.}}2つの選択肢のペアx,yに対し,$x R y \lor y R x$を満たすとき完全性を持つと言います.完全性は比較可能性とも言います.
\end{dfn}
\begin{dfn}[推移性]
    任意の二項関係$R$は,任意の3つの選択肢の組x,y,zに対して,$x R y \land y R z \to x R z$を満たすとき推移性を持つと言います.
\end{dfn}
\begin{dfn}
    弱順序とは,反射性,完全性,推移性をすべて満たす二項関係のことです.
\end{dfn}

反射性によって$x \not \geq x$のような事態は回避されます.

完全性が満たされているというのは,さきほどの4パターンのうち4番目の「どちらが良いとも言えない」というパターンはないということです.人気歌手アンケートの回答で言えば,「この人とこの人は毛色が違いすぎて優劣もなにもそもそも比較できないし,順位なんて付けられない」という回答は考えないということになります.

推移性は,「xがyより好きで,yがzより好きなら,当然xがzより好き」という自然そうな性質です.また,推移性のおかげで「コンドルセのパラドックス」の$x > y$かつ$y > z$かつ$z > x$のような選択肢間で循環が起こるおかしな選好が許されなくなります.理由は図2の例に推移性を適用ができるかを調べて確認してください.

以下に弱順序とそうでないものの例をそれぞれ図1,図2に挙げておきます.矢印「$x \leftarrow y$」は$x\geq y$の関係があることを表しています.両矢印は$\doteq$です.また,反射性は当然成り立つものとし,自分自身との関係については省略しました.
\begin{figure}[!h]
    \centering
    \begin{minipage}{0.55\hsize}
    {\scalefont{0.9}
        \begin{tikzpicture}
            \node (1) at (1,2) {桃};
            \node (2) at (0,0) {栗};
            \node (3) at (2,0) {柿};
            \draw [thick, <-] (1) -> (2);
            \draw [thick, <-] (1) -> (3);
            \draw [thick, <-] (2) -> (3);
        \end{tikzpicture}
        \begin{tikzpicture}
            \node (1) at (1,2) {桃};
            \node (2) at (0,0) {栗};
            \node (3) at (2,0) {柿};
            \draw [thick, <->] (1) -> (2);
            \draw [thick, <->] (1) -> (3);
            \draw [thick, <->] (2) -> (3);
        \end{tikzpicture}
        \begin{tikzpicture}
            \node (1) at (0,2) {犬};
            \node (2) at (2,2) {猫};
            \node (3) at (0,0) {鳥};
            \node (4) at (2,0) {魚};
            \draw [thick, ->] (1) -> (2);
            \draw [thick, <-] (1) -> (3);
            \draw [thick, <->](1) -> (4);
            \draw [thick, <-] (2) -> (3);
            \draw [thick, <-] (2) -> (4);
            \draw [thick, ->] (3) -> (4);
        \end{tikzpicture}}
        \caption{弱順序}
    \end{minipage}
    \begin{minipage}{0.4\hsize}
    {\scalefont{0.9}
        \begin{tikzpicture}
            \node (1) at (1,2) {桃};
            \node (2) at (0,0) {栗};
            \node (3) at (2,0) {柿};
            \draw [thick, <-] (1) -> (2);
            \draw [thick, ->] (1) -> (3);
            \draw [thick, <-] (2) -> (3);
        \end{tikzpicture}
        \begin{tikzpicture}
            \node (1) at (0,2) {犬};
            \node (2) at (2,2) {猫};
            \node (3) at (0,0) {鳥};
            \node (4) at (2,0) {魚};
            \draw [thick, ->] (1) -> (2);
            \draw [thick, <-] (1) -> (3);
            \draw [thick, <-] (2) -> (3);
            \draw [thick, <-] (2) -> (4);
            \draw [thick, ->] (3) -> (4);
        \end{tikzpicture}}
        \caption{弱順序でない}    
    \end{minipage}
\end{figure}

図1の一番左の三角形を見てください.この例では「$桃 > 栗 > 柿$」と選好しています.真ん中の例では「$桃\doteq 栗\doteq 柿$」と全て同率1位,右の例では「$猫>犬\doteq 魚>鳥$」で猫1位,犬・魚同率2位,鳥3位と順位がつけられます.

しかし,図2の左の例では循環してしまっています.「桃より柿,柿より栗,栗より桃が好き」などと言うものを個人的選好として扱うのは少し厳しいような気がします.また,右の例では犬と魚の順位が比較できません.このように選択肢間で比較できないとするようなものも個人的選好として許してしまうと色々と大変です.この例では実感がわきませんが,選択肢「平家物語」と「有楽町駅」で比較しろと言われても困るように,選択肢間で比較できるものだけに限るという条件は大切です.

以上の理由から,以降個人的選好・社会的選好が当然満たしている性質として弱順序を仮定することにします.特に断りがなければ$\ge,\succeq$は弱順序ですし,選好と言ったら弱順序です\footnote{と,ここでは言ってますが,5章で推移性を満たさない選好についても考えます.}.

選好を弱順序に制限することが不自然に思えるかもしれませんが,それについては第5章で少し触れることにします.

\subsection{厚生関数}
今まで選好の二項関係を用いた表し方について見てきました.次に,個人的選好から社会的選好を決めることについて考えます.個人的選好にも様々なものが考えられます.例えば,個人$i$の3選択肢$x,y,z$に対する個人的選好は,以下に示す13通りのうちのどれかになります.各自選好がこれのみに限られることを確認してください\footnote{ちなみに,n選択肢に対する個人的選好の数は$n=1$から,$1,3,13,75,541,\dots$と増えていきます.この数列を順序付きベル数,フビニ数と言います.}.
\begin{equation*}
\begin{array}[h]{cccccc}
    1 : & x > y > z &
    2 : & y > z > x &
    3 : & z > x > y \\
    4 : & x > z > y &
    5 : & y > x > z &
    6 : & z > y > x \\
    7 : & x > y \doteq z &
    8 : & y > z \doteq x &
    9 : & z > x \doteq y \\
    10: & x \doteq y > z &
    11: & y \doteq z > x &
    12: & z \doteq x > y \\
    13: & x \doteq y \doteq z & & & &
\end{array}
\end{equation*}

さて,個人$i$の個人的選好を代入できる変数を$P_i$と書くことにし,個人的選好と社会的選好を対応させる関数について考えます.

\begin{dfn}[厚生関数]
    n人の個人的選好$P_i$の組$(P_1,P_2,P_3,\dots,P_n)$に対して,社会的選好$P$を1通りに定める関数を厚生関数と言います.
\end{dfn}
例えば,最初に出てきた昼ご飯の例で,Nymwaくんの意見が採用された場合を考えると,各個人と社会の選好は,
\begin{eqnarray*}
    \begin{array}[h]{ccl}
        P_{Lupa}  &=& 牛丼 > カレー > ラーメン \\
        P_{Maxe}  &=& カレー > ラーメン > 牛丼 \\
        P_{Nymwa} &=& ラーメン > 牛丼 > カレー \\
        P &=& ラーメン \succ 牛丼 \succ カレー \\
    \end{array}\\
\end{eqnarray*}
となり,"牛丼vs.カレー"の後,"勝者vs.ラーメン"と多数決を取っていく厚生関数をfとすると,$P = f(P_{Lupa},P_{Maxe},P_{Nymwa})$と書けます.

厚生関数に代入したり出力される選好は弱順序なので,例えば$P_i$や$P$に"$x > y > z > x$"のような弱順序ではない選好が来ることはできません.また,厚生関数は必ずしも投票方式を表すわけではありません.厚生関数は個人的選好の組と社会的選好の間の対応を表す関数関係にすぎないということは理解しておいてください.

