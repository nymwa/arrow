\section{アロー4条件}
ここまで個人的選好と社会的選好の対応を表現する方法について説明しました.次に厚生関数が満たすべき条件について考えます.例えば社会的選好がデタラメに対応しているとか,ある個人の個人的選好は全く社会的選好に反映されないというような厚生関数は民主的とは言えないでしょう.一体,厚生関数にどのような条件を課せば民主的だと言えるのでしょうか.

以下に示す4つの条件は経済学者アローが「民主制には不可欠だ」として提示したものです.以降,アロー4条件と呼ぶことにします.

\subsection*{条件1.個人的選好の自由}
\begin{dfn}[個人的選好の自由]
    厚生関数の定義域(入力$(P_1, \dots, P_n)$の取りうる全選好の組の集合)が,個人的選好の可能なすべての組み合わせを含んでいる時,その厚生関数は個人的選好の自由を満たしていると言います.
\end{dfn}

文字通り,すべての個人的選好は制限されることがないという条件です.

例えば,ダム建設を推進する独裁者がいるため,誰も"$ダム建設反対 > 賛成$"という個人的選好が許されない,そんなこと言ったら秘密警察に何されるかわからん,というような社会はこの条件に反します.個人的選好の自由は,思想信条の自由が民主的な国家には不可欠であることと同様に,民主的な決定には必要なことだと考えられます.

この条件は普通,「定義域の非限定性」と呼ばれます.厚生関数の定義域は,個人的選好の組のうち考えられるどのようなものであっても良いということです.

ところで,実は選択肢の全ペアの多数決による選好の決定方法は個人的選好の自由に反しているのです.可能なすべての選択肢のペアに対して多数決を取ると社会的選好は循環して弱順序でなくなり定められないことがあるのは,最初の昼ご飯の例で示した通りです.そして,そのような個人的選好の組は対応する社会的選好を定められないため,厚生関数の定義域に属することができません.循環などが起きてこのような事態に陥った場合,誰かに選好を変えさせたり,昼ご飯の決め方を変えたりしないと何を食べるか決定できないのです.

\subsection*{条件2.社会の決定性}
\begin{dfn}[社会の決定性]
    厚生関数が定める社会的選好に関して,
    \begin{equation*}
        \forall x,y \in A (\forall i \in M (x >_i y) \to x \succ y)
    \end{equation*}
    が成り立つ時,その厚生関数には社会の決定性があると言います.
\end{dfn}

どの2選択肢$x,y$においても,すべての個人$i$に対して個人的選好が"$x >_i y$"ならば,社会的にも"$x \succ y$"となるという条件です.

これは「パレート最適性」とも呼ばれているもので,全員一致ならそれは社会的にもそうだと決められるものだという意味合いです.例えば社会の決定性のある社会で,すべての個人が「死刑廃止 $>$ 死刑存続」と選好するならば,社会的にも「死刑廃止 $\succ$ 死刑存続」と選好されなければなりません.

社会の決定性は,一見,満たされて当たり前の性質のように思えます.しかし,実際にはこの社会の決定性にはいろいろと問題があるのです.それについては第5章で述べることとして,今はこれを当然満たされるものだと思って読み進めてください.

\subsection*{条件3.無関係選択肢からの独立性}
\begin{dfn}[無関係選択肢からの独立性]
    厚生関数に"$\geq^1_i$"で表される個人的選好の組$(P^1_1,\dots,P^1_n)$を入力した結果,社会的選好が"$\succeq^1$"で表されたとします.一方,同じ個人・選択肢に対する,"$\geq^2_i$"で表される個人的選好の組$(P^2_1,\dots,P^2_n)$を入力した結果は"$\succeq^2$"となったとします.この時,
    \begin{equation*}
        \forall x,y \in A (\forall i \in M (x \geq^1_i y \leftrightarrow x \geq^2_i y) \to (x \succeq^1 y \leftrightarrow x \succeq^2 y))
    \end{equation*}
    が成り立つ時,その厚生関数には無関係選択肢からの独立性があると言います.
\end{dfn}

なんかよくわからない式が出てきました.

これは,例えば,個人的選好が途中で変化することがあったとしても,2選択肢$x,y$間のどの個人的選好も変化していないのであれば,$x,y$の社会的選好も変わらない,すなわちどの2選択肢$x,y$間の社会的選好も,$x$と$y$に関する個人的選好のみによって決まり,無関係な選択肢$z$や$w$などに関する個人的選好からは決まらないという条件です.

この条件は割とわかりにくいので,実際の投票の例を挙げて説明します.

今,ある国の次期国王を決める選挙が行なわれていることを想定しましょう.国王候補はA,B,C,D,Eの5人で,誰が国王になるかを決めるのはX,Y,Zの3人とします.この国では伝統的に,X,Y,ZがA,B,C,D,Eに対し国王にふさわしいと思う順に順位を決め,1位が5点,2位が4点,3位が3点,4位が2点,5位が1点を得点し最高点の候補が当選するという方式が使われ続けています.
この選挙方式はボルダ得点法と呼ばれています.

例えば,X,Y,Zが自分の心に忠実に,国王の素質があると思った順に順位を決めた時,図3のようになるとします.

\begin{figure}[!h]
\centering
\begin{tabular}{|c|c|c|c|} \hline
    & X & Y & Z \\ \hline
    1位 & A & A & B \\ \hline
    2位 & B & B & A \\ \hline
    3位 & C & C & C \\ \hline
    4位 & D & D & D \\ \hline
    5位 & E & E & E \\ \hline
\end{tabular}
\caption{素直に決めたとき}
\end{figure}

この場合,Aが14点,Bが13点,Cが9点,Dが6点,Eが3点で,Aが次期国王ということになります.なるはずでした.

しかし,なんと投票の前日,候補Bが投票者Zにこう持ちかけていました.
\begin{align*}
    B  & 「今のままだとAに勝てそうにない」\\
    Z  & 「私も君に勝ってほしい」\\
    B  & 「私を勝たせてくれたら大臣にしてあげるよ」\\
    Z  & 「わあい」
\end{align*}

すると,結果は図4のように変化しました.

\begin{figure}[!h]
\centering
\begin{tabular}{|c|c|c|c|} \hline
    & X & Y & Z \\ \hline
    1位 & A & A & B \\ \hline
    2位 & B & B & C \\ \hline
    3位 & C & C & D \\ \hline
    4位 & D & D & E \\ \hline
    5位 & E & E & A \\ \hline
\end{tabular}
\caption{BがZを買収したとき}
\end{figure}
なんとZがAを最下位にしてしまいました.
すると,Aが11点,Bが13点,Cが10点,Dが7点,Eが4点となり,Bが次期国王ということになってしまいます.

Zが個人的選好を変えたことで,社会的選好が変わってしまいました.しかし,よく見ると投票者$X,Y,Z$の選択肢$A,B$間の選好関係は全く変化していません.これは,選択肢A,B間の社会的選好が選択肢A,B間以外の個人的選好に影響されてしまっているということです.このことからボルダ得点法の厚生関数は無関係選択肢からの独立性を満たしていないことがわかります.

もしも候補が$A$と$B$だけならこのような事態は起きません.しかし,他の選択肢が存在したがために,Zは偽りの選好を示すことで,選挙結果を自分の都合の良いように戦略的に操作できてしまったのです\footnote{数学者ラプラスがボルダにこのことを指摘すると,ボルダは「この方式は正直者のためのものなので」と言い返したそうです.}.

無関係選択肢の独立性を満たしていない厚生関数では,自分に都合の良い選択肢が選ばれるために個人が偽りの選好をすることを推進する事態が起こりえることがわかりました.ちなみに,例えば各ペアで多数決を取る方式は無関係選択肢からの独立性を満たしているので,Zの選好の変化前後で$A$と$B$の社会的選好は変化しません.

無関係選択肢からの独立性を満たしていない厚生関数では,嘘をつかせてしまうだけでなく,選択肢の集合全体に対して投票を行った場合と,選択肢の部分集合に対して行った場合で結果が変わる可能性もあります.国王の例では,候補の部分集合$\{A,B\}$に対してボルダ得点法を用いると結果が変わってしまいます.全体の一部分だけを取り出して比較検討することもできなくなってしまうかもしれません.これは分析的な検討が意味を成さないということにつながります.正直者が馬鹿を見,選択肢間の検討よりも戦略的に選好を表明することが意味を持つこのような厚生関数が民主的と言えるでしょうか?

この条件にもまた,様々な問題があるのですが,それはまた後述します.

\subsection*{条件4.非独裁制}
\begin{dfn}[非独裁制]
    社会$M$には,厚生関数の定義域にあるすべての入力に対して,$\forall x,y \in A (x >_i y \to x \succ y)$となる個人$i$は存在しません.以降,このような個人のことを独裁者と呼ぶことにします.
\end{dfn}
どの2選択肢$x,y$についても,たとえ社会の他の人たちがどのような選好をしたとしても,ある個人が存在して「ぼくはこっちのほうがいい!」といえば社会的選好がそれと同じとなるような個人,すなわち独裁者は存在しないという条件です.この条件は明らかに民主制に不可欠だと受け入れられると思います.

