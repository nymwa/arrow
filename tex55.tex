\subsection{パレート伝染病}
社会の決定性$\forall i (x >_i y) \to x \succ y$の持つ問題は何なのでしょうか.社会の決定性を我々が「民主制に不可欠な」「自然な条件」と認識するのは,その「全員一致の判断は社会的に反映されるべきだ」という側面だけを見ているからだと言えます.社会の決定性は全員の選好の一致が起きている時に限り,無関係選択肢からの独立性,すなわち,$x$と$y$の社会的選好はその個人的選好のみから成り立つとする側面を持ちます.この無関係選択肢からの独立性を弱めたような条件が奇妙な事態を引き起こすのだと考えられます.

具体的にどのような問題が起こるのか考えます.自由主義より,ある個人$k$はある選択肢のペア$x,y$に対する決定力を持つものとします.すなわち$D_k(x\succ y) \land D_k(y \succ x)$.今,選択肢$x,y,z,w$に対して,個人の選好を図19のように,$x >_i y, z >_i w, \forall k (z >_k x), \forall k (y >_k w)$とします.

\begin{figure}[!h]
    \centering
    \begin{tabular}[!h]{|c|c|c|c|} \hline
        & i & その他 & 社会 \\ \hline
        x,y & $x > y$ &    ?    & $x \succ y$ \\ \hline
        y,w & $y > w$ & $y > w$ & $y \succ w$ \\ \hline
        z,x & $z > x$ & $z > z$ & $z \succ x$ \\ \hline
        z,w & $z > w$ &    ?    & $z \succ w$ \\\hline
    \end{tabular}
    \caption{選好}
\end{figure}

社会的選好は図19のようになりますが,表をよく見ると,個人$i$が選択肢のペア$z,w$にも決定力を持ってしまっています.これは,選択肢$y,w$,$z,x$における社会の決定性によるものです.

\begin{figure}[!h]
    \centering
    \begin{tikzpicture}
        \node (1) at (-1, 1) {x};
        \node (2) at ( 1, 1) {y};
        \node (3) at (-1,-1) {z};
        \node (4) at ( 1,-1) {w};
        \draw[thick, ->] (2) -- (1);
        \draw[thick, ->] (1) -- (3);
        \draw[thick, ->] (4) -- (2);
        \draw[ultra thick, ->] (4) -- (3);
        \node (5) at (0, 1.4) {$i$の権利};
        \node (6) at (0,-1.4) {伝染};
        \node (7) at ( 2.2,0) {全員一致};
        \node (8) at (-2.2,0) {全員一致};
    \end{tikzpicture}
    \caption{社会的選好の波及}
\end{figure}

ここで,例えば$z$が「個人$j$\footnote{iではないことに注意してください.}は一週間Twitterでヒシュカリヤナ語縛りをする」,$w$が「個人$j$は一週間Twitterでカシュヤナ語縛りをする」だとした場合,たとえ個人$j$が$w > z$と選好していても$z \succ w$となり,社会の決定性が自由主義を侵害する結果になってしまいます.このように社会の決定性(パレート最適性)と社会的選好が伝染し,個人の権利を侵害することをパレート伝染病と言います.

\subsubsection*{センの解決法}
このような自由主義のパラドックスやパレート伝染病を解決する方法の1つを紹介します.これはセンによるものなので,センの解決法と呼ぶことにします.

まず,センの解決法では,個人的選好を隠すことを許します.例えば,「実は人には言えないこういう趣味がある」などのプライベートなことを社会的に表明せずに隠しておいても良いということです.社会全体で選好が一致しても,誰か一人が選好を隠せば社会の決定性は行使されません.このような意見の保留を許す時,他人の権利が侵害されないように誰かが自分の意見を隠すことでパレート伝染病を防ぐことができます.もっと言えば,一つの社会には「他人の権利を侵害しないように自分の選好を隠す」良心的で自由主義者な個人が一人でも存在するならば,各個人が決定力を行使しても社会の決定性と矛盾しないような決定方式が存在し得るということです.今後,このような個人のことを良心的自由主義者と呼ぶことにします.

先程の「父と息子の会話」の例にこの方法を適用してみましょう.仮に父を良心的自由主義者とします.父は「就職$\succ$進学」とする決定力がありますが,これは社会的循環を発生させ,息子の「無職$\succ$就職」の権利を侵害します.そのため,父は「就職$\succ$進学」の表明を取りやめ,結局社会的には「進学$\succ$無職$\succ$就職」となります.息子が良心的自由主義者の時も同様にして,「就職$\succ$進学$\succ$無職」となります.まあ,「良心的自由主義者」などと言っていますが,この場合は息子が諦めて就職するという感じなのでしょうが.また,ここで誰が良心的自由主義者となるのかという問題も生じますが,それ以上は社会的倫理などの領域になるかと思われますし,考察を止めたいと思います.

今までの議論で見てきたように,社会の決定性には選択肢の中に含まれる個人の権利を侵害することがあります.選択肢と言って我々が普通イメージするのは,例えば「本を一冊選んでください」という状況での,「資本論」「国富論」「幸福論」などのようなものです.しかし,選択肢が対象とするものは単なるモノだけではなく,個人の権利や社会的に意味を持つ個人の選択なども含まれて来てしまいますし,そのような選択肢が存在しうることについても考慮しなければなりません.

このような社会的な選択は,単に選択肢間の順位付けをするだけでなく,その決定における倫理性というものをも規定することができます.これに関しては詳しいことは専門的な図書を当たるなどして調べていただきたいです.社会の決定性が引き起こしうる問題点について理解し,後述する架空世界の投票方式の考察でより深い理解を得るための参考にしていただければよいかと思われます.

